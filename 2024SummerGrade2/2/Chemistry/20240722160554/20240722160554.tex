
\documentclass[10pt,cn]{elegantbook}
\usepackage[utf8]{inputenc}
\usepackage[T1]{fontenc}
\usepackage{tgtermes}
\usepackage{amsmath}
\usepackage{amsfonts}
\usepackage{amssymb}
\usepackage{stmaryrd}
\usepackage{hyperref}
\hypersetup{colorlinks=true, linkcolor=blue, filecolor=magenta, urlcolor=cyan,}
\urlstyle{same}
\usepackage{graphicx}
\usepackage[export]{adjustbox}
\usepackage{mdframed}
\usepackage{booktabs,array,multirow}
\usepackage{esint}
\usepackage{xeCJK}
\usepackage{adjustbox}
%\graphicspath{ {./images/} }

\usepackage{ulem}
\usepackage{hyperref}%目录跳转

\usepackage{fontspec} % 用于处理字体
%\setmainfont{TeX Gyre Termes} % 设置主要字体


\usepackage{fancyhdr}   % 导入 fancyhdr 包,用于定制页眉和页脚
\usepackage{datetime}   % 导入 datetime 包,用于格式化日期

%\fancyhead[L]{20240722} % 左侧页眉
%\fancyhead[R]{\mydate\today} % 右侧页眉,显示当前日期,格式为“日 月 年”

\title{HappaChemistryNotes}
\subtitle{化学笔记}
\author{OyamaHappa}
\date{\today}
\version{20240721222406}
\logo{logo.jpg}
\cover{cover.jpg}



% 本文档命令
\usepackage{array}
\usepackage{mathdots}
\newcommand{\ccr}[1]{\makecell{{\color{#1}\rule{1cm}{1cm}}}}
% 修改目录深度
\setcounter{tocdepth}{3}

\everymath{\displaystyle}%用行间公式(displaystyle)的格式排版所有的行内公式


%\usepackage{verbatim}%在codeshow中已引用
\usepackage{tikz,tkz-euclide}
\usepackage{amsmath}
\usepackage{pgfplots}
%\usepackage{codeshow}%codeshow:为了在codeshow环境中,引用代码,并生成图形。

\usepackage{graphicx}
%\usepackage{subfigure}

\usepackage{breqn}%breqn 宏包主要提供了 dmath 和 dmath* 等几个环境,产生可以自动折行的显示公式。
\usepackage{longtable}%长表格,多页可以自动处理。

%【参与编译的文件列表。】
%\includeonly{preface,chapter01,chapter02,chapter03,chapter04,chapter05}%,%【参与编译的文件列表。】




\usepackage{amsmath}
\usepackage{amsfonts}
\usepackage{amssymb}
\usepackage{stmaryrd}
\usepackage{hyperref}
\hypersetup{colorlinks=true, linkcolor=blue, filecolor=magenta, urlcolor=cyan,}
\urlstyle{same}
\usepackage{graphicx}
\usepackage[export]{adjustbox}
\usepackage{mdframed}
\usepackage{booktabs,array,multirow}
\usepackage{esint}
\usepackage{xeCJK}
\usepackage{adjustbox}
\newcommand{\HRule}{\begin{center}\rule{0.5\linewidth}{0.2mm}\end{center}}
%\graphicspath{ {./images/} }
\usepackage{amsmath}
\usepackage{pifont}

\begin{document}
	
	\maketitle
	
	\tableofcontents
	%\listofchanges
	
	\mainmatter

	\part{化学反应原理}
    \chapter{中和滴定实验}
	\section{滴定实验}
	
	我们在研究物质时, 常常需要对物质进行定性分析和定量分析。确定物质的成分, 包括元素、 无机物所含的离子和有机物所含的官能团等, 在化学上叫做定性分析。测定物质中元素、离子、官能团等各成分的含量, 在化学上叫做定量分析。
	
	%【思考】现有一瓶未知浓度的 \(\mathrm{{NaOH}}\) 溶液,如何准确测出其浓度?
	
	\subsection{酸碱中和滴定}
	
	利用中和反应原理, 用已知物质的量浓度的酸 (或碱) 来测定未知物质的量浓度的碱 (或酸) 的方法
	
	1、中和反应:
	
	2、中和滴定原理:
	
	\[
	\mathrm{{HA}} + \mathrm{{BOH}} = \mathrm{{BA}} + {\mathrm{H}}_{2}\mathrm{O}
	\]
	
%	\(1\mathrm{\;{mol}}\;1\mathrm{\;{mol}}\)
	
	即可得 \({\mathrm{c}}_{\left( \mathrm{{HA}}\right) }{\mathrm{V}}_{\left( \mathrm{{HA}}\right) } = {\mathrm{c}}_{\left( \mathrm{{BOH}}\right) }{\mathrm{V}}_{\left( \mathrm{{BOH}}\right) }\)
	
	现在我们用 \({0.1000}\mathrm{\;{mol}}/\mathrm{L}\) 的 \(\mathrm{{HCl}}\) 溶液测定未知浓度的 \(\mathrm{{NaOH}}\) 溶液,到底应测得哪些数据才能求出 \({\mathrm{c}}_{\left( \mathrm{{NaOH}}\right) }\) ?
	
	\begin{center}
		\includegraphics[max width=0.5\textwidth]{image/c1.jpg}
	\end{center}
	
	\subsection{滴定实验仪器以及操作要点}
	
	\subsubsection{滴定方法的关键}
	
	(1)准确测定两种反应物溶液的体积
	
	(2)确保标准液、待测液浓度的准确
	
	(3)\uline{滴定终点的准确判定}(包括指示剂的合理选用)
	\subsubsection{实验仪器及试剂}
	
	\begin{center}
		\includegraphics[max width=0.2\textwidth]{image/c2-1.jpg}
	\end{center}
	
	(1)仪器:酸式滴定管、碱式滴定管、滴定管夹、烧杯、锥形瓶、铁架台。
	
	\begin{center}
		\includegraphics[max width=0.2\textwidth]{image/c2-2.jpg}
	\end{center}
	\[\mbox{酸式滴定管}\]
		\[\mbox{润洗--用待装液体洗涤}\]
	\begin{center}
		\includegraphics[max width=0.2\textwidth]{image/c2-3.jpg}
	\end{center}
		\[\mbox{碱式滴定管}\]
	 
	
	\subsubsection{滴定管的构造特点}
	
		\ding{172}标识: 标有温度、刻度、规格(25.00 mL 或 \({50.00}\mathrm{\;{mL}}\) )%圈1
	
	\ding{173}刻度:零刻度在\_\_\_,满刻度在\_\_\_;最小刻度为 0.1 ml,精确度为 0.01 ml。%圈2
	
	\ding{174}酸式滴定管: 下端是玻璃塞, 能盛装 \_溶液;
	
	碱式滴定管: 下端是橡皮管+玻璃小球, 能盛装\_
	
	%【思考 1】现在要量取 \({25.00}\mathrm{\;{mL}}\mathrm{{NaOH}}\) 溶液,应该用什么量取? 能不能用量筒? 为什么?
	
%	【思考 2】如图是某些仪器的刻度部分示意图, 图中各仪器虚线为所示读数。其中为量筒的是 \_\_\_(填编号,下同),读数为 \(2\text{、}n\mathrm{\;L}\) ;为滴定管的是\_\_\_,读数为 \(2\text{、}\mathrm{{SmL}}\)
	
%	\begin{center}
	%	\includegraphics[max width=0.3\textwidth]{images/0190d453-f338-754b-8cea-9ab8ce034fb0_0_605307.jpg}
	%\end{center}
	
%	① \ding{173}
	\subsubsection{凡士林的涂抹方式}
	
	\begin{center}
		\includegraphics[max width=0.2\textwidth]{image/c3-1.jpg}
	\end{center}
	
		\[\mbox{涂a和c}\]
	
	
	润洗仪器: 在加入酸、碱之前, 洁净的酸式滴定管和碱式滴定管要分别用所要盛装的酸、碱润洗 \(2 \sim 3\) 次。
	
	\ding{173}方法是: 从滴定管上口加入 \(3 \sim 5\mathrm{\;{mL}}\) 所要盛装的酸溶液或碱溶液。倾斜着转动滴定管,使液体润湿全部滴定管内壁。然后, 一手控制活塞 (轻轻转动酸式滴定管的活塞; 或者轻轻挤压碱式滴定管中的玻璃球),将液体从滴定管下部放入预置的烧杯中。
	
	\ding{174}加入反应液: 分别将酸溶液、碱溶液加到酸式滴定管、碱式滴定管中,\uwave{ 使液面位于滴定管刻度“0”以上 \(2 \sim 3\mathrm{\;{mL}}\) 处},并将滴定管垂直固定在滴定管夹上。
	
	
	\ding{175}调节起始读数: 在滴定管下放一个烧杯, 调节活塞, 使滴定管尖嘴部分充满反应液, 并使液面处于“0”刻度(或“0”刻度以下),准确读取读数并记录$¥V_{\mbox{始}}$
	
	【思考 7】如果滴定管内出现气泡怎么排出气泡? 
	
	排气泡: 酸式滴定管 \(\rightarrow\)尖嘴部分朝上,碱式滴定管\(\rightarrow\)
	
	\begin{center}
		\includegraphics[max width=0.2\textwidth]{image/c4-1.jpg}
	\end{center}
		\[\mbox{除去碱式滴定管乳胶管中气泡的方法}\]
	
	
	\ding{176} 放液
	
	a 从碱式滴定管中放出 \({25.00}\mathrm{{ml}}\) 氢氧化钠溶液于锥形瓶中;
	
	
	
	\(\mathrm{b}\) 滴入几滴\uwave{酚酞试液}(指示剂),将锥形瓶置于酸式滴定管下方,并在瓶底衬一张白纸。
	
		\ding{177}滴定: 左手控制酸式滴定管活塞, 右手摇动锥形瓶, 边滴入盐酸 (当接近终点时, \uwave{改为滴加半滴酸}) , 边不断顺时针方向摇动, 眼睛要始终注视
	
	\begin{center}
		\includegraphics[max width=0.3\textwidth]{image/c4-2.jpg}
	\end{center}
	
	\ding{178}记读数: $\star$ \uwave{当滴入最后半滴HCl,溶液由红色突变为无色,且半分钟内不褪色. 停止滴定,准确记下盐酸读}数 \({\mathrm{V}}_{\text{终}}\) ,并准确求得滴定用去的盐酸体积 \(\mathrm{V} = {\mathrm{V}}_{\text{始 }} - {\mathrm{V}}_{\text{终 }}\) (平行实验 2 -3 次) 
	
	滴入最后半滴标准溶液具体操作?
	
	\ding{179}计算
	
	【经典 2】【2020 年 7 月选考】滴定前, 有关滴定管的正确操作为 (选出正确操作并按序排列,选项可重复使用): 检漏 \(\rightarrow\) 蒸馏水洗涤 \(\rightarrow\) 用滴定液润洗 2 至 3 次 \(\rightarrow\) 装入滴定液至零刻度以上 \(\rightarrow\) 排除气泡 \(\rightarrow\) 调整滴定液液面至零刻度或零刻度以下\( \rightarrow \)记录起始读数 \(\rightarrow\) 开始滴定。
	
	
\subsection{指示剂的选择}

\subsubsection{酸碱指示剂}

(1)酸碱指示剂的变色范围(pH 值) 

\begin{center}
	\adjustbox{max width=\textwidth}{
		\begin{tabular}{|c|c|c|c|}
			\hline
			\multirow{2}{*}{甲基橙} & \(< {3.1}\) & \({3.1} \sim {4.4}\) & \(> {4.4}\) \\
			\hline
			& 红 & 橙 & 黄 \\
			\hline
			\multirow{2}{*}{酚酞} & \(< {8.2}\) & \({8.2} \sim {10}\) & \(> {10}\) \\
			\hline
			& 无色 & 浅红 & 红 \\
			\hline
			\multirow{2}{*}{石蕊} & \(< 5\) & \(5 \sim 8\) & \(> 8\) \\
			\hline
			& 红 & 紫 & 蓝 \\
			\hline
		\end{tabular}
	}
\end{center}
\paragraph{滴定终点}

\[\mbox{显酸}\Rightarrow\mbox{甲基橙}\]
\[\mbox{显碱}\Rightarrow\mbox{酚酞}\]
\paragraph{变化曲线}

若以酸碱中和滴定过程中滴加酸 (或碱) 的量为横轴,以溶液的 \(\mathrm{{pH}}\) 为纵轴,即可绘出的一条溶液 \(\mathrm{{pH}}\) 随酸(或碱)的滴加量而变化的曲线。

\begin{center}
	\includegraphics[max width=0.3\textwidth]{image/c5.jpg}
\end{center}

\subsection{小结}

\subsubsection{指示剂的选择原则}

①变色要明显、灵敏;

②指示剂的变色范围要尽可能在滴定过程中的 \(\mathrm{{pH}}\) 值突变范围内。

③指示剂用量不能太多, \(2 \sim 3\) 滴即可:

\subsubsection{指示剂的选择(由滴定曲线可知)}

①强酸强碱相互滴定, 可选用甲基橙或酚酞。

②若反应生成强酸弱碱盐, 溶液呈酸性, 则选用酸性变色范围的指示剂 (甲基橙);

若反应生成强碱弱酸盐, 溶液呈碱性, 则选用碱性变色范围的指示剂 (酚酞)

③石蕊试液因颜色变化不明显, 且变色范围过宽, 一般不作滴定指示剂。

④酸性 \({\mathrm{{KMnO}}}_{4}\) 溶液等本身呈现颜色的滴定试剂,不用另外选择指示剂

\subsubsection{终点判断}

 滴入最后半滴 XX 标准溶液后, 溶液由 XX 色突变 XX 色, 且半分钟内不褪色。

\begin{center}
	\adjustbox{max width=\textwidth}{
		\begin{tabular}{|c|c|c|}
			\hline
			指示剂 操作 & 酚酞 & 甲基橙 \\
			\hline
			强碱滴定强酸 & 无色变为红色 & 橙色变为黄色 \\
			\hline
			强酸滴定强碱 & 红色变为无色 & 黄色变为橙色 \\
			\hline
		\end{tabular}
	}
\end{center}

\subsection{误差分析}

以一元酸和一元碱的中的滴定为例

\[
{\mathrm{C}}_{\text{待 }}{\mathrm{V}}_{\text{待 }} = {\mathrm{C}}_{\text{标· }}{\mathrm{V}}_{\text{标 }}
\]

滴定过程中任何错误操作都有可能导致 \(\mathrm{C}\) 标、 \(\mathrm{V}\) 标、 \(\mathrm{V}\)待 的误差。但在实际操作中认为 \(\mathrm{C}\) 标是已

知的, \({\mathrm{V}}_{\text{待 }}\) 是固定的,对于 \({\mathrm{c}}_{\left( \mathrm{{NaOH}}\right) }  = \frac{{\mathrm{c}}_{\left( \mathrm{{HCl}}\right) }{\mathrm{V}}_{\left( \mathrm{{HCl}}\right) }}{{\mathrm{V}}_{(NaOH)} } 
$

\[\mbox{读数比实际}\]

\begin{center}
%	\adjustbox{max width=\textwidth}{
		\begin{tabular}{|c|c|c|c|c|}
			\hline
			\phantom{X} & 产生误差的常见因素 & & 对 \(\mathrm{V}_{\mathrm{HCl}}\) 的影响 & 对 \(\mathbf{C}_{\mathrm{NaOH}}\) 的影响 \\
			\hline
			\multirow{4}{*}{滴定前操作} & 未用标准液润洗酸式滴定管 & [HCL]$\downarrow$& \(\uparrow\) & \(\uparrow\) \\
			\cline{2-5}
			& 未用待测液润洗碱式滴定管 &[NaOH]$\downarrow$& \(\downarrow\) & \(\downarrow\) \\
			\cline{2-5}
			& 用待测液润洗锥形瓶 &NaOH$\uparrow$& \(\uparrow\) & \(\uparrow\) \\
			\cline{2-5}
			& 洗涤后锥形瓶未干燥 &n(NaOH)不变& \(-\) & \(-\) \\
			\hline
			\multirow{2}{*}{滴定时读数不准} & 滴定前俯视酸式滴定管,滴定后平视 & & \(\uparrow\) & \(\uparrow\) \\
			\cline{2-5}
			& 滴定前仰视酸式滴定管,滴定后俯视&  & \(\uparrow\) & \(\uparrow\) \\
			\hline
			\multirow{2}{*}{取液时读数不准} & 取待测液时先俯视后仰视& & \(\downarrow\) & \(\downarrow\) \\
			\cline{2-5}
			& 取待测液时先仰视后俯视&  &\(\uparrow\) & \(\uparrow\) \\
			\hline
			\multirow{6}{*}{操作不当} & 滴定前酸式滴定管有气泡,滴定后气泡消失&  & \(\uparrow\) & \(\uparrow\) \\
			\cline{2-5}
			& 滴定前酸式滴定管无气泡,滴定后有气泡&  & \(\downarrow\) & \(\downarrow\) \\
			\cline{2-5}
			& 滴定结束,滴定管尖端挂一滴液体未滴下 & & \(\uparrow\) & \(\uparrow\) \\
			\cline{2-5}
			& 滴定过程中,振荡锥形瓶时,不小心将溶液溅出 & & \(\uparrow\) & \(\uparrow\) \\
			\cline{2-5}
			& 用甲基橙作指示剂,滴至橙色,半分钟内又还原成黄色,不处理就计算 & & \(\uparrow\) & \(\uparrow\) \\
			\cline{2-5}
			& 配制标准液的固体有不反应的杂质 & & \(\downarrow\) & \(\uparrow\) \\
			\hline
		\end{tabular}
%	}
\end{center}


\section{其他滴定}

\subsection{氧化还原滴定}

\subsubsection{酸性 \({\mathrm{{KMnO}}}_{4}\) 溶液滴定 \({\mathrm{H}}_{2}{\mathrm{C}}_{2}{\mathrm{O}}_{4}\) 溶液}

原理: \(2{\mathrm{{MnO}}}_{4}^{ - } + 6{\mathrm{H}}^{ + } + 5{\mathrm{H}}_{2}{\mathrm{C}}_{2}{\mathrm{O}}_{4} = {10}{\mathrm{{CO}}}_{2} \uparrow + 2{\mathrm{{Mn}}}^{2 + } + 8{\mathrm{H}}_{2}\mathrm{O}\) ;

指示剂及滴定终点: 酸性 \({\mathrm{{KMnO}}}_{4}\) 溶液本身呈紫红色,不用另外选择指示剂,当滴入最后半滴酸性 \({\mathrm{{KMnO}}}_{4}\) 溶液,溶液由无色变浅红色,且半分钟内不变色,说明达到滴定终点。

\subsubsection{ \({\mathrm{{Na}}}_{2}{\mathrm{\;S}}_{2}{\mathrm{O}}_{3}\) 溶液滴定碘液}

原理: \(2{\mathrm{\;S}}_{2}{\mathrm{O}}_{3}^{2 - } + {\mathrm{I}}_{2} = {\mathrm{S}}_{4}{\mathrm{O}}_{6}^{2 - } + 2{\mathrm{I}}^{ - }\) ;

指示剂及滴定终点: 用淀粉溶液+作指示剂,当滴入最后半滴 \({\mathrm{{Na}}}_{2}{\mathrm{\;S}}_{2}{\mathrm{O}}_{3}\) 溶液,溶液的蓝色褪去, 且半分钟内不恢复原色, 说明达到滴定终点。


\chapter{原电池}

\section{原电池}

\textit{将化学能转化为电能的装置, 本质为自发进行的氧化还原反应。}

\paragraph*{正极}

化合价$\downarrow$,得$e^{-}$$\Rightarrow$牺牲阳极的阴极保护法

\paragraph*{负极}

化合价$\uparrow$,失$e^{-}$,氧化反应

\subsection{原电池的构成条件}

1、活性不同的两极 
2、自发的氧化还原反应
3、闭合回路 
4、电解质

\section{盐桥}

\begin{center}
	\includegraphics[max width=0.5\textwidth]{image/c14.jpg}
\end{center}

\subsection{盐桥构成}

盐桥里的物质一般是\dotuline{强电解质}而且不与两池中电解质反应, 常使用装有饱和 \(\mathrm{{KCl}}\) 琼脂溶胶的 \(\mathrm{U}\) 形管,离子可以在其中自由移动。

盐桥作用:

电极反应方程式: 

三大流向: 

\section{膜电池}

膜的引入简化了装置, 用离子交换膜分隔成两池, 仅允许特定的离子通过; 且膜能持续、长期使用。

\begin{center}
	\includegraphics[max width=0.9\textwidth]{image/c16-1.jpg}
\end{center}

膜的分类:
 \textit{叫什么就只让什么离子过}

\ding{172}阳离子交换膜%圈1

\ding{173}阴离子交换膜

\ding{174}质子交换膜

只有$H^{+}$能过

\ding{175}双极膜

$H_{2}O\rightleftharpoons H^{+}+OH^{-}$一人去一边

\section{电解池 1}

\subsection{电解池}

\begin{center}
	\includegraphics[max width=0.2\textwidth]{image/c19.jpg}
\end{center}

%请从电解水的能量变化和物质变换, 分析电解池的原理。

%\section*{电解池}

\textit{电解池: 把电能转变为化学能的装置。}

\paragraph{构成条件}

1.电源

2.两个电极(只导电,可用惰性电极--石墨,铂Pt,金Au)

3.电解质(水/熔融)

4.闭合回路

\paragraph{两个电极}

\paragraph{流向}

\subparagraph*{电子}

阳$\rightarrow$阴

\subparagraph*{电流}

阴$\rightarrow$阳

\subparagraph*{$\star$离子}

异性相吸(阳离子$\rightarrow$阴极)

\end{document}